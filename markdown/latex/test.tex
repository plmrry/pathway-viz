% !TEX TS-program = xelatex
% !TEX encoding = UTF-8 Unicode

% -----------------
% START OF PREAMBLE
% -----------------
\documentclass[12pt,a4paper]{scrreprt}


% Commands
\newcommand{\HRule}{\rule{\linewidth}{0.5mm}}


% Packages
\usepackage{eurosym}
\usepackage{amssymb}
\usepackage{mathtools}
\usepackage{upquote}
\usepackage{microtype}
\usepackage{polyglossia}
\usepackage{longtable,booktabs}
\usepackage{graphicx}
\usepackage{grffile}
\usepackage[normalem]{ulem}
\usepackage[setpagesize=false,
            unicode=false,
            colorlinks=true,
            urlcolor=blue,
            linkcolor=black]{hyperref}


% Polyglossia settings
\setmainlanguage{english} % or danish
\addto\captionsenglish{%
  \renewcommand{\contentsname}{Table of Contents}
}
\addto\captionsdanish{%
  \renewcommand{\contentsname}{Indholdsfortegnelse}
}


% Required for syntax highlighting


% Don't let images overflow the page
% Can still explicit set width/height/options for an image
\makeatletter
\def\maxwidth{\ifdim\Gin@nat@width>\linewidth\linewidth\else\Gin@nat@width\fi}
\def\maxheight{\ifdim\Gin@nat@height>\textheight\textheight\else\Gin@nat@height\fi}
\makeatother
\setkeys{Gin}{width=\maxwidth,height=\maxheight,keepaspectratio}


% Make links footnotes instead of hotlinks


% Avoid problems with \sout in headers with hyperref:
\pdfstringdefDisableCommands{\renewcommand{\sout}{}}


% No paragraph indentation
% and set space between paragraphs
\setlength{\parindent}{0pt}
\setlength{\parskip}{1em plus 2pt minus 1pt}
\setlength{\emergencystretch}{3em}  % prevent overfull lines


% -----------------
%  END OF PREAMBLE
% -----------------
\begin{document}

% chapter: 0, section: 1, subsection: 2 etc
\setcounter{secnumdepth}{1}
\setcounter{tocdepth}{1}
\tableofcontents

hello

\chapter{Section}\label{section}

This is our first section.

Another paragraph.

We can use Markdown for figures.

Markdown for lists

\begin{itemize}
\tightlist
\item
  One
\item
  Two

  \begin{itemize}
  \tightlist
  \item
    Nested one
  \end{itemize}
\item
  Three

  \begin{enumerate}
  \def\labelenumi{\arabic{enumi}.}
  \tightlist
  \item
    Numerated list
  \item
    No need to specify number
  \end{enumerate}
\end{itemize}

We can even inline math: \(y = ax + b\).
How about displayed equations:

\[
y = -2.2x + 0.5
\]

\section{Subsection}\label{subsection}

Just use Markdown to define sections and structure of the document.

Let's finish with a footnote.\footnote{I'm a footnote!}

\end{document}
